\documentclass{jfp1}
\title{The GHC Runtime System}

\author{fill in}

\begin{document}

\maketitle

\begin{abstract}
(This paper describes the implementation of the GHC runtime system)
\end{abstract}

\tableofcontents

\section{Introduction}

The GHC runtime system is perhaps one of the most unusual language
runtimes in wide use today.  Much of its implementation was directly
motivated by the unusual (by popular programming language standards!)
features in Haskell which the runtime needs to support.  Some of these
features, e.g. lazy evaluation and ubiquitous concurrency, complicate
the design of the system and require us to do a lot more work to give
programmers the high-level behavior they desire.  Other features, such
as an emphasis on pure computation without mutation, simplify the design
components such as garbage collection and software transactional memory.

\section{Storage}

An essential component of a runtime system for any high-level
programming language is the garbage collector, which is responsible
identifying and reclaiming memory from objects which are no longer in
use by the program.

The presence of a garbage collector imposes important restrictions on
the design of the runtime system.  Additionally, the implementation of a
garbage collector is greatly complicated by its performance
considerations: the garbage collector affects the performance of all
programs.  Fortunately, the design of the garbage collector itself has
very few dependencies on the rest of the runtime system (the most
interesting thing about our setting is the unusual data life cycle that
accompanies functional programs, which we will discuss in Section~\XXX).

Let us assume, then, that we are building a runtime for a programming
language with managed memory.  What do we need?

\subsection{Blocks}

The very first consideration is this: ``Where is the memory coming
from?''  The conventional design is to request a large, contiguous block
of memory from the operating system and use it as the heap.  However,
this scheme is fairly inflexible, especially when multiple heaps are
necessary (as might be the case in a generational garbage collector.)

Instead, GHC utilizes a \emph{block-structured heap}~\cite{Marlow:2008:PGG:1375634.1375637}.
The basic idea is to divide the heap into fixed-size $B$-byte blocks,
where $B$ is a power of two: blocks are then linked together in order to
provide memory for the heap.  GHC uses 4kb blocks, but this is an easily
adjustable constant.  With blocks, we can allocate and dynamically
resize regions of memory at will.  There are some other benefits as well:

\begin{enumerate}
    \item Large objects (e.g. close to block size) do not have to be copied from one region to
        another; instead, the block they reside in can be relinked from
        one region to another.\footnote{Of course, this requires \emph{only}
        one object to live in a block, which can result in fragmentation.
        However, empirically this does not seem to have caused much of a problem.}
    \item Blocks make it easy to provide heap memory in contexts where it is
        not possible to perform garbage collection.  The primary example of this
        in GHC is the GMP arbitrary-precision arithmetic library, which requires
        the ability to allocate memory for its internal computation.  If the heap
        was contiguous, we would be left with no recourse if the heap ran out
        of memory while GMP code was executing, since there is no way to resize
        the heap without simultaneously executing a garbage collection, and
        precise garbage collection cannot be performed safely while running C code.
    \item Free memory can be recycled quickly, since a free block can be quickly
        used somewhere else.
\end{enumerate}

One reason why this scheme works so well is that most objects on the
heap are much smaller than the block size; handling these cases is very
simple.  When an object is larger than a block size, it needs to be
placed into a \emph{block group} of contiguous blocks---which in turn
need to be handled with some care to avoid fragmentation.  The blocks
themselves are provided by the operating system in units called
\emph{megablocks}, which in the current implementation are 1Mb in size.

Finally, each block has associated with a \emph{block descriptor}, which
contains information about the block such as what generation it belongs to, how full it is, what block
group it is part of, etc.  An obvious place to put the block descriptor
is at the beginning of a block, but this runs into problems when the block
is the member of a block group (the memory must be contiguous!)
Thus, the descriptors of blocks of a megablock are instead organized together
in the first block of a megablock; some care is taken to ensure that the
runtime can efficiently compute the block descriptor of any given block---an
interested reader can find the details at \verb|http://hackage.haskell.org/trac/ghc/wiki/Commentary/Rts/Storage/BlockAlloc|
\XXX

\subsection{Memory layout}

The next classic problem a garbage collector encounters is the question:
``What is a pointer?''  The garbage collector must be able to answer
this question in order to follow the pointers and get an accurate
picture of what memory is in use.  The classic solution to this problem
is to require data on the heap to be \emph{boxed}: every object has
a header which says what type of object it is and what the pointer fields
in the object are.

GHC adopts this design choice:  every object has a \emph{header} which
points to an \emph{info table}.  The info table indicates what the type of
the object is and what the pointer fields of the object are.  GHC supports
two types of layout: one where the payload consists of zero or more pointers
followed by zero or more non-pointers (pointers-first), and another where
pointers are described by a bitmap.

\Red{This bit is a mess}

Unusually, the info table also contains \emph{code} for the object: any
object can be ``entered'' in order to evaluate it.  This may seem
wasteful for a data constructor, which has no evaluation to do (and
simply returns immediately).  However, for Haskell, which features
lazy evaluation, allowing all objects to be evaluatable makes a good
deal of sense: there is now no longer any need to make a memory dereference
to find out whether or not some data is a thunk or a data constructor: just jump
to it. \Red{But see pointer tagging.}

\verb|http://hackage.haskell.org/trac/ghc/wiki/Commentary/Rts/Storage/HeapObjects|

\subsection{Generational garbage collection}

\subsection{Parallel garbage collection}

\subsection{Summary}

\section{Concurrency and parallelism}

We now turn our attention to the implementation of
concurrency~\cite{PeytonJones:1996:CH:237721.237794} and
parallelism~\cite{Harris:2005:HSM:1088348.1088354} in the GHC runtime.
\Red{Something here; maybe the old saw about the difference between
concurrency and parallelism, or maybe something about why Haskell's
concurrency/parallelism support is so cool.}

\Red{Some motivation here}

%   The implementation of many of the features in this section can be
%   divided into two parts: an implementation in the uniprocessor case
%   (concurrency), and an implementation in the multiprocessor case
%   (parallelism).

\subsection{Threads}

Concurrent Haskell~\cite{PeytonJones:1996:CH:237721.237794} exposes the
abstraction of a \emph{Haskell thread}, which is a concurrent thread of
execution.  Given that operating systems also native threads, the
essential question for an implementation of Haskell threads is this:
what is the correspondence between Haskell threads and OS threads?

A simple scheme is to require a \emph{one-to-one} mapping, an approach
taken by many other languages with multithreading support.
Unfortunately, this approach is expensive: the GHC runtime would like to
support thousands of threads, which is supported poorly by most
underlying operating systems.  Furthermore, in the case that a program
is run with only one operating system thread (e.g. the code in question
is not thread safe), only one thread of execution is supported: all
concurrency is lost, and must be implemented in userspace, as is the
case with many asynchronous IO libraries.

Instead, we must \emph{multiplex} multiple Haskell threads onto a single
OS thread.  This requires two adjustments: first, we must be able to
induce compiled Haskell code to yield to a scheduler, so that another
Haskell thread can be run.  It is very easy for a thread to
cooperatively suspend itself: all it needs to do is ask for a garbage
collection; thus, we can preempt a thread by setting its heap limit to
zero, triggering a faux ``heap overflow'' that transfers control to the
scheduler.\footnote{This can have problems when a thread is in a tight,
    non-allocating loop; however, better concurrency in such cases can
be achieved by forcing the compiler to emit heap checks at all function
prologues, even when no allocation is necessary.}
Second, we must save the thread's context, e.g. the stack pointer and
the errno flag, so that it can be restored when the thread is
rescheduled.  Because the compiled code is responsible for yielding to
the scheduler, it can be very efficient about how much other processor
state must be saved; while general purpose C-based coroutine libraries
must save all registers,~\XXX{} a Haskell thread will usually only need
one or two registers to be saved.\footnote{This is what the fast path \texttt{stg\_gc\_*} functions are for.}

Once we have a way of suspending and resuming threads, the scheduler
loop is quite simple: maintain a run queue of threads, and repeatedly:

\begin{enumerate}
    \item Pop the next thread off of the run queue,
    \item Run the thread until it yields or is preempted,
    \item Check why the thread exited:
        \begin{itemize}
            \item If it ran out of heap, call the GC and then run the thread again;
            \item If it ran out of stack, enlarge the stack and then run the thread again;
            \item If it was preempted, the thread is pushed to the end of the queue;
            \item If the thread exited, continue.
        \end{itemize}
\end{enumerate}

Once we have a scheduler loop, we have a simple way of running multiple
Haskell threads on a \emph{single} OS thread, even when the Haskell
threads are not thread-safe.

\subsection{Foreign function interface}

While multiplexing is a very attractive solution for supporting
lightweight concurrency, there are some places when the illusion of ``a
Haskell thread is simply an operating system thread, but cheaper''
breaks down.  These cases are especially prevalent when considered with
the \emph{foreign function interface} (FFI)~\cite{Marlow04extendingthe},
which permits Haskell code to call out to foreign code not compiled by
GHC, and vice versa.  One particular problem is as follows: what if an
FFI call blocks?  As our concurrency is purely cooperative, if the FFI
call refuses to return to the scheduler, the execution of all other
threads will grind to a halt.

The upshot is that we need to decouple OS threads into two parts: the OS
thread itself (called a \emph{Task} in GHC terminology), and the Haskell
execution context (called a \emph{Capability}).  The Haskell execution
context can be thought of as the scheduler loop and is responsible for
the contents of the run queue; when executing it is owned by the
particular OS thread which is running it.  A single-threaded Haskell
program, there would only be one capability---in this case, the
capability can be thought as the lock on the Haskell runtime.  This
decoupling allows a capability to be run on different operating system
threads as necessary: now a blocking FFI call can be managed by
releasing the capability before making the FFI call: if there is another
idle \emph{worker thread}, it can pick up the now free capability and
continue running Haskell code.  It is preferable to move the Haskell
execution context, rather than the FFI call, to another thread, as it
avoids an OS level thread context switch.\footnote{To see this, consider
the execution from the perspective of the OS thread, which executes
an FFI call by releasing a lock on an object (the capability),
running some C code, and then re-acquiring a lock on the
capability.}

Thus, we have to modify the scheduler loop as follows:

\begin{enumerate}
    \item Check if we need to yield the capability to some other OS thread, e.g. if an FFI call has just finished,
    \item \emph{Run as before.}
\end{enumerate}

Another problem that occurs is when the FFI code maintains some thread
local state.  When we pass capabilities around OS threads, we make no
guarantee that any given FFI call will be performed on the same OS
thread: this can cause problems for FFI code of this type.  To
accomodate this, Haskell introduces the concept of a \emph{bound
thread}, which is a thread which is guaranteed to run on the same
operating system thread whenever it is scheduled.  GHC refers to these
as \emph{in-calls}, because any external code which calls into GHC will
always create a bound thread, as its Haskell code may make an FFI call
which must be scheduled on the same original OS thread.

It is simple to modify the scheduler so that no other thread may use
an OS thread that a thread is bound to, but we can do better.  We must modify
the scheduler loop some more:

\begin{enumerate}
    \item \emph{Check if we need to yield the capability to some other OS thread, e.g. if an FFI call has just finished,}
    \item \emph{Pop the next thread off of the run queue,}
    \item Check if the thread is bound:
        \begin{itemize}
            \item If the thread is bound but is already scheduled on the OS thread, proceed.
            \item If the thread is bound but on the wrong OS thread, give the capability to the correct task.
            \item If the thread is not bound but this OS thread is bound, give up the capability, so that any capability that truly needs this OS thread will be able to get it.
        \end{itemize}
    \item \emph{Run as before.}
\end{enumerate}

The movement of capabilities from task to task is somewhat intricate,
but it is necessary to properly support FFI calls, and imposes very
little overhead when no FFI calls are being made.

\subsection{Load balancing}

Assuming that the compiled Haskell code is thread safe~\XXX, it is now
very simple to have the scheduler loop run on multiple OS threads:
allocate multiple capabilities!  Each OS thread in possession of a capability
runs the scheduler loop, and everything works the way you'd expect.

There is one primary design choice: should each capability have its own
run queue, or should there be a single global run queue?  A global run
queue avoids the need for any sort of load balancing, but requires
synchronization and makes it difficult to keep a Haskell thread on one
core, destroying data locality.  With separate run queues, however,
threads must be load balanced, lest one capability accumulate too many
threads while the other capabilities idle.

The very simple load balancing scheme GHC uses is as follows: when a capability
runs out of threads to run, it suspends itself (releasing its lock) and
waits on a condition variable.  When a capability has too many threads
to run (it checks each iteration of its schedule loop), it takes out
locks on as many idle capabilities as it can, and then pushes its excess
threads onto their run queues.  Finally, it releases the locks and
signals on each idle capability's condition variable that they can
continue running.  The benefit of this scheme is that the run queues are
kept completely unsynchronized. \Red{Why isn't a work stealing queue used here?}

\subsection{Messages and whiteholes}

Up until now, we focused purely on how to ensure multiple threads could
execute concurrently, without considering the other essential aspect of
concurrency: synchronized communication.  Haskell offers a variety of
ways for Haskell threads to interact with each other (MVars,
asynchronous exceptions, STM, even lazy evaluation).  When these features
are implemented on a uniprocessor, they can be implemented directly;
in a multithreaded setting, some synchronization is necessary.

Under the hood, the GHC runtime has two primary ways of synchronizing:
\emph{messages} and \emph{whiteholes} (effectively spinlocks).  The
runtime makes very sparing use of OS level condition variables and
mutexes, since these operations tend to be expensive.

GHC uses a very simple message passing architecture to pass messages
between capabilities.  A capability sends a message by performing the following steps:

\begin{enumerate}
    \item Allocate a message object on the heap with the appropriate data;
    \item Take out a lock on the message inbox of the destination capability;
    \item Append the message onto the inbox;
    \item Interrupt the capability, using the same mechanism as the context switch timer (setting the heap limit to zero);
    \item Release the lock.
\end{enumerate}

This allows the message to be handled by the destination capability at
its convenience, i.e. after the running Haskell code yields and we
return to the scheduling loop.  The primary benefit of message passing
is it removes the need to synchronize any of the non-local state that a
capability \emph{might} want to modify. \Red{Why don't the message
inboxes use spinlocks?}

When a closure on the heap needs to be locked, e.g. because it is a
mutable cell, GHC uses a very simple spinlock, attempting to replace the
header of a closure with a \emph{whitehole} header.  If another thread
enters the closure, they will spinlock until the original header is
replaced. \Red{An example of the entry code indirection helping out!}
Spinlocks are great for locking closures, since the critical regions
they protect are very short, and it would be expensive to allocate a
mutex for every closure that needed one.

\subsection{MVar}

We now describe how to implement MVars, the simplest form of
synchronization available to Haskell threads.  An MVar is a mutable
location that may be empty.  There are two operations which operate on
an MVar: \verb|takeMVar|, which blocks until the location is non-empty,
then reads and returns the value, leaving the location empty, and
\verb|putMVar|, which dually blocks until the location is empty, then
writes its value into location.

The blocking behavior is the most interesting aspect of MVars:
ordinarily, one would have to implement this using a condition variable.
However, because our Haskell threads are not operating system threads,
we can do something much more lightweight: when a thread realizes it
needs to block, it adds itself to a \emph{blocked threads queue}
corresponding to the MVar, and then removes itself from the run queue.
When another thread fills in the MVar, it can check if there is a thread
on the blocked list and wake it up immediately.  In a multithreaded
setting, the only necessary synchronization is a spinlock for the queue
manipulation and a wakeup message to the capability which owns the
unblocked thread.

This scheme has a number of good properties.  First, it allows us
to implement efficient \emph{single wake-up} on MVars, where only one of
the blocking threads is permitted to proceed. Second, using a FIFO
queue, we can offer a fairness guarantee, which is that no thread
remains blocked on an MVar indefinitely unless another thread holds the
MVar indefinitely.  Finally, because threads are garbage collected
objects, if the MVar a thread is blocking on becomes unreachable,
\emph{so does the thread.}  Thus, in some cases, we can tell when
a blocked thread will never make any further progress, and terminate it.

\subsection{Asynchronous exceptions}

MVars are a cooperative form of communication, where a thread must
explicitly opt-in to receive messages.  Asynchronous
exceptions~\cite{Marlow:2001:AEH:378795.378858}, on the other hand
permit threads to asynchronously raise an exception in another thread
without its cooperation.  Asynchronous exceptions are much more
difficult to program than their synchronous brethren: as the signal can
occur at any point in a program's execution, the program must be careful
to register handlers which ensure that any locks are released and
invariants are preserved.  In \emph{pure} functional programs, this requirement
is easier to fulfill, as pure code can always be safely aborted.  When
asynchronous exceptions are available, however, they are useful
in a variety of cases, including timing out long running computation,
aborting speculative computation and handling user interrupts.

With messages, it is very simple to trigger an asynchronous exception:
simply send a throw-to message to the capability which owns the thread
to be interrupted.  Once the thread is preempted, the scheduler induces
an exception and walks up the stack, much in the same way as a normal
exception would be handled.

There are two primary differences.  The first is that a thread which
is messaged may have been blocking on some other thread (i.e. on an MVar);
thus, when an asynchronous exception is received, the thread must remove itself
from the blocked list of threads.  This is usually relatively simple, although
if your queues are singly linked you will need some cleverness to remove
a thread from it.\footnote{GHC does this by stubbing out an entry with an
indirection, the very same that is used when a thunk is replaced with its
true value, and modifying queue handling code to skip over indirections.}

The second difference is how lazy evaluation is handled. When pure code
raises an exception, referential transparency demands that any other
execution of that code will result in the same exception.  Thus, while
we are walking up the stack, when we see update frames \Red{link back},
we replace the thunk it would have updated with a new thunk that always
the exception.  In the case of an asynchronous exception, however, the
code could have simply been unlucky: when someone else asks for the same
computation, we should simply resume where we left off.  Thus, when we
hit the first update frame, we instead record the entire stack above the
frame on to the thunk (an \verb|AP_STACK|).  Evaluating an
\verb|AP_STACK| concatenates the old stack fragment to the current
stack, thus resuming the computation.  Finally, replace the current top
of the stack (originally an update frame) with a pointer to the
\verb|AP_STACK| and continue recursively. \Red{Check the details here with RaiseAsync.c:raiseAsync}

\subsection{STM}

Software Transactional Memory, or STM, is an abstraction for concurrent
communication which emphasizes transactions as the basic unit of
communication.  The big benefit of STM over MVars is that they are
composable: while programs involving multiple MVars must be very careful
to avoid deadlock, programs using STM can be composed together
effortlessly.

Before discussing what is needed to support STM in the runtime system,
it is worth mentioning 

\subsection{Sparks}

Up until now, the threads we have discussed were explicitly asked for
by the user.  In some cases, the user will only have one thread (because
their application does not need to be concurrent), in which case extra
cores will be unused.  Is there any way we can take advantage of extra
capacity?

Haskell takes advantage of pure lazy evaluation to offer \emph{sparks},
which are extremely lightweight threads with the express purpose of
evaluating a thunk to head normal form.  Because thunks do not have
side effects, it is safe to evaluate them speculatively, or to not
evaluate them at all; if they are truly needed they will get forced
appropriately.

Since sparks are even lighter-weight than threads, they are represented
separately and stored in \emph{spark pools}.  When a capability decides
it has no work to do, it creates a \emph{spark thread}, which repeatedly
attempts to retrieve a spark from the capability's spark pool and
evaluate it.  If the capability receives any real work (e.g. a thread
unblocks), then it responds to the spark threads request that there are no
more sparks to evaluate, so that it exits.

It is relatively important to ensure sparks can be load balanced across
capabilities, as the capability that is most likely to be generating
sparks is one that also is doing real work.  Sparks are balanced using
bounded work-stealing queues~\XXX; empirically, it seems that using a FIFO
ordering (so the oldest sparks are stolen first) is best, although this
requires more synchronization on our implementation of work-stealing
queues.

\Red{Comment about interaction with GC}

\subsection{Synchronization on the heap}

In previous sections, we asserted that execution of Haskell threads
could be made parallel, assuming that the compiled Haskell code was
thread-safe.  As any developer who has needed to write thread-safe code
can attest, this is a tall order!  However, in Haskell we have a leg up:
the vast majority of code written is pure and does not perform explicit
mutation, so we do not mind paying the costs of explicitly synchronizing
other mutable state (as is the case for MVars). However, there is one
major performance bottleneck that we have to worry about: lazy
evaluation!  Recall that after a thunk has been evaluated, its value on
the heap must be replaced with the fully evaluated value.  As multiple
threads could evaluate a thunk in parallel, these writes must be
synchronized. A poorly implemented synchronization scheme could be
extremely costly.

\Red{describe the scheme in \cite{Harris:2005:HSM:1088348.1088354}}

\subsection{Summary}

\subsection{Further reading}

\section{Execution model}

We begin our exploration of GHC's runtime system with a description
of the execution model of compiled Haskell code.

While the execution model of compiled code isn't in the domain of the
runtime system per se (it is the compiler's responsibility to generate
code that abides by the various conventions), various parts of the
execution model (e.g. heap representation, lazy evaluation) have an
important role to play in the design of components of the runtime system
(e.g. the garbage collector, concurrency).  In this section, we offer a
brief explanation of some of the most important aspects of the
\emph{Spineless Tagless G-machine} (STG), GHC's core execution model.

\subsection{G-machine}

How does a Haskell program run?  This is not a straightforward question
to answer, because a compiled Haskell program is not run directly: it is
transformed from Haskell into an intermediate representation called
Core, optimized, and transformed into another intermediate
representation (STG) before any code is generated.  Describing how
Haskell is desugared and what these optimizations are is out of scope
for this paper.  We have to discuss 

This intermediate representation is the STG, which has the benefits of
being both a functional language and having a clear operational reading.

What are the basic operations that STG code supports?  It 

\begin{itemize}
    \item Function application,
    \item Literals,
    \item Constructor application (saturated),
    \item Primitive operations (saturated),
    \item Evaluation of expressions and case-split of the result,
    \item Heap allocation (let) of closures (thunks or functions) and constructors,
    \item Let-no-escape,
\end{itemize}


\subsection{(Untitled)}

What does compiled Haskell code do?  Haskell is a high-level language,
and thus the executable code produced by a Haskell compiler bears little
resemblance to the code that was originally written.

Fortunately, GHC does have a \emph{purely
functional} intermediate language, which does have an operational
reading: this language is the \emph{Spineless Tagless
G-machine}.~\cite{PeytonJones1992}.

STG is 

\begin{itemize}
    \item Function application: tail call (direct jump)
    \item Let expression: heap allocation
    \item Case expression: evaluation of a thunk
    \item Constructor application: return to continuation
\end{itemize}



The fact that all function applications are tail calls is perhaps what
causes the largest divergence between the runtime execution of Haskell
and other block-structured languages.  In C, when a function is entered
a stack frame (e.g. continuation) is always pushed onto the stack; in
Haskell, we may push zero, one, or more frames on the stack before
jumping to the function destination.

While there are many details to STG, for which an interested reader
should consult \cite{PeytonJones1992} and \cite{Marlow2006}, a brief
overview of the STG will be useful for understanding some of the things
the runtime system needs to support.  STG is a simple lambda calculus,
for which the following common operations are given an operational
reading:

\subsection{Spineless Tagless G-machine}

The Spineless Tagless G-machine (STG) is 

\subsection{Heap representation}

All data on the heap is boxed, represented with a \emph{header}, which
indicates what kind of object the data is, and the \emph{payload}, which
contains the actual data for an object.  The header points to an
\emph{info table}, which provides more information about whether or not
the object is a thunk, a data constructor or a function, and describes
the layout of the payload in more detail (e.g. what fields are
pointers.) \Red{Figure}

\subsection{Spineless: Stack}

\Red{Maybe defer this later?}

\Red{Haskell code executes utilizing a \emph{eval/apply} model}



\subsection{Tagless: Lazy evaluation}

Unusually, the info table also contains \emph{code} for the object: any
object can be ``entered'' (by jumping to the code in the infotable) in
order to evaluate it.  This is perhaps the most important design
decision in STG: it prescribes a uniform data representation for both
fully evaluated data values (constructors and functions) and unevaluated
thunks---referred to collectively as \emph{closures}.  In the original
design of the STG, all code wanting to access a field in a data
structure first entered the closure and upon return would receive the
values of the data structure (a \emph{vectored return}).  The lack of
any check whether or not a thunk was evaluated or not (an obvious
alternative implementation strategy) is behind the ``tagless`` in STG's
name.

\SM{We don't do vectored returns, since GHC 4.0. Also
  return-in-registers, which you might also find menioned in the
  original STG paper, was removed in favour of unboxed tuples.}

\SM{Tagless is a bit debatable these days.  We have tags in pointers
  (pointer tagging).  Also the code pointer for a function closure is
  the entry code for the function (not the ``eval'' code).}

The fact that objects are always entered is an extremely helpful layer
of indirection which is used for a variety of purposes by the runtime.
For example, when a thunk finishes evaluation, we need to write back the
true value to the old memory location so the computation is not
repeated.  If the new value is larger than the thunk, it must be placed
in new heap memory and the old thunk replaced with an \emph{indirection}
pointer pointing to the new value.  An indirection object, then, simply
jumps to the code of the indirectee!  This flexibility is useful for a
number of other exceptional conditions, especially in the case of
concurrency.

Unfortunately, code compiled this way performs a lot of indirect jumps,
and modern branch prediction units on processors typically perform very
poorly under such control flow.  This might suggest that this data
representation is quite expensive and of little interest to implementors
of non-lazy languages.  Fortunately, modern GHC implements a
\emph{dynamic pointer tagging} scheme~\XXX{} which, in many cases,
eliminates the need to perform an indirect jump.  This scheme works by
using the lower order bits (two bits in a 32-bit machine, and three bits
in a 64-bit machine) in order to encode whether or not pointer is
already evaluated, and if it is, what the tag of the constructor is.
When the tag bit is absent, user code will enter the closure, as before.

We think that this representation is well worth considering even for
non-lazy-by-default languages.  In the case of data types with multiple
constructors, pointer tagging enables us to support efficient case
analysis of user-defined types \emph{without a memory
dereference}---only the tag bit must be consulted.  And, to reiterate,
the added flexibility the indirect jumps will greatly aid us later in
other components of the runtime system.

\Red{Maybe explain in more detail}

\subsection{Black holes} \label{sec:blackhole}

One particular type of closure worth some attention is the \emph{black
hole}.  The semantics of a black hole are relatively simple: a black
hole represents a thunk that is currently being evaluated.  A thunk
can be \emph{claimed} by overwriting it with a black hole. The entry
code for a black hole should arrange for the thread to receive the value
of the thunk after it is evaluated, some way or another.

Black holes were originally proposed as a solution for a space leak that
occurs when tail calls are being made in the STG.~\cite{Jones2008} The
problem is relatively simple: suppose that you are evaluating the thunk \verb|last [1..10000]|,
where \verb|last| is a tail recursive function:

\begin{verbatim}
last []     = error "empty list"
last (x:[]) = x
last (x:xs) = last xs
\end{verbatim}

Because \verb|[1..10000]| is constructed lazily, you might expect
evaluating this thunk to take constant space, since the front of the
list is never retained.  However, there is a problem: the front of the
list is retained by the thunk itself, \verb|last [1..10000]|.  The thunk
will eventually get overwritten by the result of the computation, but
only at the \emph{end} of the computation, by which point the entirety
of \verb|[1..10000]| is resident in memory.  To solve this problem, when
\verb|last [1..10000]| is evaluated, we can \emph{eagerly} overwrite it
with a \verb|BLACKHOLE|; now the thunk no longer retains the list.  As
an optimization, we can \emph{lazily} blackhole by waiting until the
thread becomes descheduled, e.g. for a garbage collection.

The simplicity of black holes belies their utility in a variety of contexts.
In particular:

\begin{itemize}
\item Black holes allow us to detect some infinite loops: if a thread attempts
    to evaluate a black hole which it previously claimed, that is an infinite loop!
\item In a multiprocessor setting, black holes allow us to avoid duplicating work when multiple
    threads attempt to evaluate the same object.\footnote{It is worth noting that for this use-case, we have to \emph{eagerly} blackhole thunks.}  Instead, a thread blocks on the owner
    of a black hole, waiting for it to finish processing.  This is described in more
    detail in Section~\ref{sec:sync}.
\end{itemize}

\subsection{Notes}

\Red{More stuff}

While the original paper about the Spineless Tagless
G-machine~\cite{PeytonJones1992} remains the best source for an
in-detail explanation about STG, many details have changed over the two
decades it has been last published.  If one had to sum up its modern
implementation in GHC, one might call it the ``Usually-Spineless
Mostly-Tagged G-machine.''  Briefly, the important changes are as
follows:

\begin{itemize}
    \item STG is no longer compiled into C, instead, it is
        compiled into a more low-level language C--~\cite{Jones1999}, allowing
        GHC to change details such as calling conventions and utilize
        more low level functionality (e.g. tail calls, explicit stack
        layout, computing targets of jumps and implementing exceptions).
    \item The stipulation that closures have a uniform representation
        has been relaxed.  The most important change is how functions
        are represented: while the paper originally proposed a push/enter
        model, GHC now uses an eval/apply model~\cite{Marlow2006}, in
        which the code pointer of function info tables points to the code
        for the function itself, and not code for reading arguments off of
        the stack.  This also means that the stack representation is
        different: there is now only a single stack for continuations and
        update frames (alongside the second C stack for register spills).
    \item Furthermore, pointers are dynamically tagged with information
        that may allow code to avoid jumping into a closure in order to
        ensure that it is evaluated---STG is not
        tagless!~\cite{Marlow2007}  The original dynamic tagging paper
        suggests that these tags could be erased for only a performance
        hit, but in fact, in some places, they are required.  This also
        means that vectored-returns have gone the way of the dinosaur.
    \item The spineless in STG's name refers to the fact that GHC stores
        the intermediate state of evaluating a thunk on the stack,
        rather than on the ``spine'' of the thunk itself.  Under some
        circumstances, such as in the case of an interrupt, it is
        profitable to write out this state back to the thunk, so that
        this work can be resumed later.~\cite{Reid1999} \Red{Maybe this is gratuitous}
\end{itemize}


\end{document}
